\def\R{\mathbf{R}}
\def\N{\mathbf{N}}
\def\L{\mathbf{L}}
\def\Z{\mathbf{Z}}

\def\bcirc{\mathbin{\mathpalette\makebcirc{\circ}}} 
\def\bast{\mathbin{\mathpalette\makebcirc{\ast}}}
\def\msurr{\mathsurround=0pt}
\def\makebcirc#1#2{% #1 je <style primitive> a #2 \circ nebo \bullet
   \ooalign{$#1\bigcirc\msurr$\cr \hfil$#1#2\msurr$\hfil}}

\def\corr{\bast}
\def\conv{\ast}





\ods{Digitalizace}


 \begin{itemize}
 \end{itemize}

 
 
 \begin{itemize}
 \end{itemize}



\ods{Konvoluce}

\begin{equation}
(f\conv g)(x)=\int f(t)g(x-t)\d t
\end{equation}


\vecobr{0.6}{konvoluce}{K definici konvoluce v 1D.}
ve 2D.}

\ods{Korelace}


\begin{equation}
(f\corr g)(x)=\int f(t)g(t-x)\d t
\end{equation}


\begin{equation}
F(u)=\int f(x)e^{-2\pi i u x}\d x
\end{equation}


\begin{equation}
\mathcal{B}=\{\sin mx,\cos nx,1\}
\end{equation}
 
\begin{equation}
F(u,v)=\int f(x,y)e^{-2\pi i (ux+vy)}\d x\d y
\end{equation}

\begin{equation}
f(x)=\int F(u)e^{2\pi i u x}\d u
\end{equation}

\begin{equation}
f(x,y)=\int F(u)e^{2\pi i (u x+vy)}\d u\d v
\end{equation}
 
\begin{eqnarray}
f_1+f_2\stackrel{CFT}{\rightarrow} F_1+F_2\\
\alpha f \stackrel{CFT}{\rightarrow} \alpha F
\end{eqnarray}


$\int \delta(x)=1$.

\begin{eqnarray}
\delta(x)&=&\left\{\begin{array}{ll}0\quad x\neq0\\.\quad x=0\end{array}\right.\\
\int\limits_{-\infty}^\infty\delta(x)\d x&=&1
\end{eqnarray}


\begin{eqnarray}
f\conv\delta&=&f\quad\forall f\\
FT(\delta)&=&1\\
FT(1)&=&\delta
\end{eqnarray}


\begin{eqnarray}
FT(f\conv g)&=&F\cdot G\\
FT(f\cdot g)&=&F\conv G
\end{eqnarray}




\begin{eqnarray}
f(x)&=&\left\{\begin{array}{ll}c\quad\forall x\in\langle-\lambda,\lambda\rangle,c\in \R\\
0\quad\hbox{jinde}\end{array}\right.\\
\noalign{\hbox{pak}\nonumber}\\
F(u)&=&\int\limits_{-\infty}^{\infty}f(x)e^{-2\pi iux}\d x=c\int\limits_{-\lambda}^\lambda e^{-2\pi i u x}\d x=\nonumber\\
&=&\frac{c}{-2\pi i u}\left[e^{-2\pi uix}\right]_{-\lambda}^\lambda=c\frac{e^{-2\pi iu\lambda}-e^{2\pi iu\lambda}}{-2\pi i u}=\nonumber\\
&=&\frac{c}{\pi u}\sin(2\pi\lambda u)\\
\noalign{\hbox{Pro $c=2$ a $\lambda=\frac{1}{2}$ bude\nonumber}}\\
F(u)&=&\frac{\sin(\pi u)}{\pi u}=\mathrm{sinc}(u)
\end{eqnarray}






\begin{equation}
f(x-a)\stackrel{CFT}{\rightarrow} e^{-2\pi i a u} F(u)
\end{equation}
\begin{eqnarray}
f(x-a)&=&g(x)\stackrel{CFT}{\rightarrow}\int g(x)e^{-2\pi i u (x-a+a)}\d x=\nonumber\\
&=&\int f(x-a)e^{-2\pi i u (x-a)-2\pi i u a}\d x=\nonumber\\
&=&e^{-2\pi i u a}\int f(x-a)e^{-2\pi i u (x-a)}\d x=\bigg\{\hbox{subst. $y=x-a$}\bigg\}=\nonumber\\
&=&e^{-2\pi i u a}\int f(y)e^{-2\pi i u y}\d y=e^{-2\pi i u a}F(u)
\end{eqnarray}



\subsection{Digitalizace}

\begin{equation}
\end{equation}



\begin{eqnarray}
s(x,y)&=&\sum\limits_{i=-\infty}^\infty\sum\limits_{j=-\infty}^\infty\delta(x-i\Delta x,y-j\Delta y)\\
d(x,y)&=&f(x,y)\cdot s(x,y)
\end{eqnarray}

\begin{eqnarray}
S(U,V)&=&\frac{1}{\Delta x}\frac{1}{\Delta y}\sum\limits_{i=-\infty}^\infty\sum\limits_{j=-\infty}^\infty
\delta(\frac{u-i}{\Delta x},\frac{u-j}{\Delta y})\\
D(u,v)&=&F(u,v)\conv S(u,v)\label{shannon}
\end{eqnarray}


\begin{itemize}
\end{itemize}

 
\begin{equation}
\mathrm{sinc}(x,y)=\frac{\sin(\pi x)\sin(\pi y)}{\pi x \pi y}
\end{equation}


\begin{enumerate}
\end{enumerate}



\begin{itemize}
\end{itemize}



\medskip

\begin{tabular}{ll}
\end{tabular}

\medskip

\subsection{Operace s Histogramem}

\begin{description}
\end{description} 



\begin{enumerate}
\end{enumerate}

\subsection{Image enhancement methods} 
\ods{Jas a kontrast}





\begin{equation}
f=f_{\mathrm{org}}+n
\end{equation}




 
\begin{eqnarray}
n\corr n&=&\sigma^2\cdot \delta\stackrel{CFT}{\rightarrow}\sigma^2\label{korel}\\
N\cdot N^\star&=&|N|^2
\end{eqnarray}




\begin{enumerate}
\begin{eqnarray}
C&=&\frac{1}{9}\left(\begin{array}{ccc}1&1&1\\1&1&1\\1&1&1\end{array}\right)
\end{eqnarray}
nebo
\begin{eqnarray}
C&=&\frac{1}{16}\left(\begin{array}{ccc}1&2&1\\2&4&2\\1&2&1\end{array}\right)
\end{eqnarray}




Tato metoda


\begin{eqnarray}
w=\frac{1}{4}(0,\dots,0,1,2,1,0,\dots,0)
\end{eqnarray}
\end{enumerate}





\begin{equation}
\end{array}\right.
\end{equation}

\begin{eqnarray}
C&=&\frac{1}{8}\left(\begin{array}{ccc}1&1&1\\1&0&1\\1&1&1\end{array}\right)
\end{eqnarray}







jako
\begin{eqnarray}
SNR=\frac{|N|^2}{|F|^2}(u,v)
\end{eqnarray}


\begin{eqnarray}
SNR=\frac{|N|^2}{|F|^2}(u,v)=\frac{\sigma_n^2}{\sigma_f^2}.
\end{eqnarray}


\begin{eqnarray}
snr=-10\log\frac{\sigma_n^2}{\sigma_f^2}
\end{eqnarray}


\subsubsection{Aproximace}


\begin{eqnarray}
min[\lambda \sum\limits_{ij} \|f(x_{ij})-u_{ij}\|^2+\mathcal{J}(f)]\\
\noalign{\hbox{kde}\nonumber}\\
\mathcal{J}(f)=\int\!\!\!\!\int\left(\frac{\partial^2f}{\partial x^2}\right)^2+2\frac{\partial^2f}{\partial x\partial y}+
\left(\frac{\partial^2f}{\partial y^2}\right)^2
\end{eqnarray}


\begin{eqnarray}
min[\lambda \sum\limits_{ij}w_{ij}\|f(x_{ij})-u_{ij}\|^2+\mathcal{J}(f)],
\end{eqnarray}


\subsection{Detekce hran}

\begin{description}
\end{description}

funkce intenzity.


\begin{enumerate}
\begin{eqnarray}
C=\left(\begin{array}{ccc}1&0&-1\\1&0&-1\\1&0&-1\end{array}\right)
\end{eqnarray}



\begin{eqnarray}
C=\left(\begin{array}{ccc}-1&-2&-1\\0&0&0\\1&2&1\end{array}\right)
\end{eqnarray}


\begin{eqnarray}
C_1&=&\left(\begin{array}{ccc}1&1&1\\1&1&1\\1&1&1\end{array}\right)\\
C_2&=&\left(\begin{array}{ccc}-1&-2&-1\\0&0&0\\1&2&1\end{array}\right)\\
C_3&=&C_1\circ C_2=\left(\begin{array}{ccc}-&-&-\\0&0&0\\+&+&+\end{array}\right)\\
(f\conv C_1)\conv C_2&=&f\conv C_3
\end{eqnarray}



\begin{eqnarray}
\Delta (G\conv f)=(\Delta G)\conv f
\end{eqnarray}

Laplaceovy rovnice $\Delta f=0$

\begin{eqnarray}
\Delta =\left(\begin{array}{ccc}0&1&0\\1&-4&1\\0&1&0\end{array}\right)
\end{eqnarray}

\begin{eqnarray}
h =\left(\begin{array}{ccc}-1&1&1\\-1&-2&1\\-1&1&1\end{array}\right)
\end{eqnarray}

\begin{itemize} 
\item jedna hrana -- jedna odezva
\end{itemize}

Postup:

\begin{enumerate} 
$$\left|\begin{array}{cc}1&0\\0&-1\end{array}\right|$$
\end{enumerate}

\end{enumerate}





\begin{eqnarray}
C&=&\left(\begin{array}{ccc}0&-1&0\\-1&5&-1\\0&-1&0\end{array}\right)
\end{eqnarray}





\begin{itemize}
\item[a)]
\begin{enumerate}
\item provedeme segmentaci obrazu
\end{enumerate}

\begin{equation}
C_B=\frac{f_{xx}f_{yy}-f_{xy}^2}{(1+f_x^2+f_y^2)^2}
\end{equation}


\begin{equation}
\phi(x,y)=\tan^{-1}\left(\frac{f_y}{f_x}\right)
\end{equation}


\begin{eqnarray}
C_{KR}&=&\frac{(\phi_x,\phi_y)(-f_y,f_x)}{\left|(-f_y,f_x)\right|}\left|(f_x,f_y)\right|=\nonumber\\
&=&(\phi_x,\phi_y)(-f_y,f_x)=\frac{f^2_xf_{yy}-2f_xf_yf_{xy}+f_y^2f_{xx}}{f_x^2+f_y^2}
\end{eqnarray}



\begin{equation}
K=\frac{\overline{f^2_x}\,\overline{f^2_y}-(\overline{f_x}\,\overline{f_y})^2}{\overline{f^2_x}+\overline{f^2_y}}
\end{equation}

\end{itemize}


\subsection{Restaurace obrazu}

\begin{eqnarray}
\mathcal{O}&=&T_G\circ T_I\\
g&=&T_G\circ T_I (f) +n
\end{eqnarray}


\begin{eqnarray}
g&=&f\conv h +n \label{eqv1}
\end{eqnarray}



\begin{eqnarray}
G&=&F\cdot H\label{eqv2}
\end{eqnarray}

\begin{eqnarray}
F&=&G/H\label{eqv3}\\
\hat{f}&=&\mathcal{FT}^{-1}(F)
\end{eqnarray}




\begin{eqnarray}
g&=&f \conv h+n\label{eqv4}\\
G&=&F\cdot H+N\label{eqv5}\\
F&=&\frac{G-N}{H}=\frac{G}{H}-\frac{N}{H}\label{eqv6}
\end{eqnarray}




\begin{eqnarray}
H&=&\left(\begin{array}{ccc}a&b&a\\b&c&b\\a&b&a\end{array}\right)\\
G&=&F\cdot H
\end{eqnarray}


\begin{eqnarray}
\hat{F}&=&G\cdot W\\
\noalign{\hbox{kde $\hat{F}$ se definuje}\nonumber}\\
\hat{F}&=&\|f-\hat{f}\|^2\\
W&=&\mathrm{argmin}E(\|f-\hat{f}\|^2)\label{minimalizace}
\end{eqnarray}


\begin{eqnarray}
W&=&\frac{1}{H}\cdot\frac{|H|^2}{|H|^2+\phi(u,v)}
\end{eqnarray}

\begin{eqnarray}
\phi&=&\frac{|N|^2}{|F|^2}
\end{eqnarray}


\begin{eqnarray}
\end{eqnarray}

\begin{eqnarray}
\phi&=&\frac{\sigma_n^2}{\sigma_f^2}
\end{eqnarray}



\begin{itemize}
\end{itemize}

\begin{itemize}
jas. 
\end{itemize}


\begin{itemize}
\end{itemize}




\begin{itemize}
  \begin{eqnarray}
    H\sim \frac{J_1}{r}\label{bessel}\qquad\begin{array}{ccl}\hbox{$J_1$}&\hbox{\dots}&\hbox{Besselova funkce 1. druhu}\\
    \end{array}
  \end{eqnarray}

\end{itemize}

\begin{itemize}
\end{itemize}


\begin{itemize}
\item omezenost supportu $h$
\end{itemize}

\begin{eqnarray}
g_1&=&f\conv h_1+n_1\\
&\vdots&\nonumber\\
g_l&=&f\conv h_l+n_l
\end{eqnarray}



\begin{eqnarray}
g=T_G(f)
\end{eqnarray}




\begin{itemize}
\end{itemize}



\begin{itemize} 
\end{itemize}



\begin{eqnarray}
u&=&f(x,y)\\
v&=&g(x,y)\\
u_i&=&f(x_i,y_i)\\
\end{eqnarray}

\begin{itemize}
\end{itemize}


\begin{itemize}

\begin{eqnarray}
\frac{E[(X-EX)(Y-EY)]}{varX varY}
\end{eqnarray}


\begin{eqnarray}
C(X,Y)=\frac{\sum\limits_{i=0}^{N-1}
\sum\limits_{j=0}^{N-1}((X_{ij}-\bar{X})\cdot(Y_{ij}-\bar{Y}))}
{\sqrt{\sum\limits_{i=0}^{N-1}\sum\limits_{j=0}^{N-1}(X_{ij}-\bar{X})^2\cdot(Y_{ij}-\bar{Y})^2}}
\end{eqnarray}



\begin{eqnarray}
f\corr g \stackrel{CFT}{\rightarrow}F\cdot G^\star
\end{eqnarray}


transformaci bude

\begin{eqnarray}
f(x)&\stackrel{CFT}{\rightarrow}&F(u)\\
g(x)&\stackrel{CFT}{\rightarrow}&e^{iau}F(u)\\
\frac{F\cdot G^\star}{|F||G|}&=&\frac{F^2e^{iau}}{|F|^2}\stackrel{CFT^{-1}}{\longrightarrow}\delta(x-a)
\end{eqnarray}


 kde vztah $\sum|f_{ij}-g_{ij}|$
\end{itemize}



\begin{itemize}




 \begin{itemize}
 \end{itemize}
\end{itemize}


\begin{eqnarray}
u&=&a_0+a_1x+a_2y\nonumber\\
v&=&b_0+b_1x+b_2y\label{afinita}
\end{eqnarray}


\begin{eqnarray}
u&=&\frac{a_0+a_1x+a_2y}{1+c_1x+c_2y}\nonumber\\
v&=&\frac{b_0+b_1x+b_2y}{1+c_1x+c_2y}\label{perspekt}
\end{eqnarray}




\begin{eqnarray}
u&=&a_0+a_1x+a_2y+a_3xy\ [{}+a_4x^2+a_5y^2\ ]\nonumber\\
v&=&b_0+b_1x+b_2y+b_3xy\ [{}+b_4x^2+b_5y^2\ ]
\end{eqnarray}

Definujme funkci $d$ 

\begin{eqnarray}
d&=&\sum\limits_{i=1}^K (f(x_i)-y_i)^2\label{minkvad}
\end{eqnarray}

\begin{eqnarray}
\frac{\partial d}{\partial a_i}&=&0\qquad\forall i\in \hat{K}
\end{eqnarray}


\begin{itemize}

\end{itemize}


\begin{eqnarray}
u=\sum a_i\phi_i(r_i)
\end{eqnarray}





\begin{enumerate}
\begin{itemize}
\end{itemize}
\begin{itemize}
\end{itemize}
\end{enumerate}



\begin{itemize}
\end{itemize}


\ods{Jak konstruhovat nadplochy?}





\ods{NN -- Nearest neighbour}
\begin{itemize}
\end{itemize}


\begin{tabular}{ccc}
$\omega_1,\dots,\omega_n$ &\dots& individua\\
\end{tabular}




\begin{eqnarray}
Cov(\xi,\nu)&=&E(\xi-E\xi)(\nu-E\nu)\\
Cor(\xi,\nu)&=&\frac{Cov(\xi,\nu)}{\sqrt{D\xi D\nu}}
\end{eqnarray}


\begin{equation}
p(\omega_i|x)=\frac{p(x|\omega_i)p(\omega_i)}{\sum\limits_j p(x|\omega_j)p(\omega_j)}
\end{equation}

\begin{description}
\end{description}



\def\x{\mathbf{x}}
\def\m{\mathbf{m}}
\def\bSigma{\mathbf{\Sigma}}

\begin{eqnarray}
p(x)&=&\frac{1}{\sigma \sqrt{2\pi}}e^{-\frac{(x-\mu)^2}{2\sigma^2}}\\
p_N(\x)&=&\frac{1}{(2\pi)^\frac{N}{2}\sqrt{|\bSigma|}}e^{-\frac{1}{2}(\x-\m)^T\bSigma^{-1}(\x-\m)}
\end{eqnarray}


Pro $N=2$ bude
$$
\Sigma=\left(\begin{array}{cc}\sigma_1^2 & Cov\\ Cov & \sigma_2^2\end{array}\right)
$$






\begin{itemize}
\end{itemize}


Maximum likelihood test:
\begin{equation}
\max\limits_q\prod\limits_{k=1}^N p(x_k|q)
\end{equation}


\begin{eqnarray}
e&=&\frac{\sum x_i}{N}\\
\sigma^2&=&\frac{1}{N}\sum(x_i-e)^2
\end{eqnarray}



\begin{itemize}
\end{itemize}

jsou:
\begin{itemize}
\end{itemize}




\begin{eqnarray}
p(\omega_i|x)&=&\max\limits_i\prod\limits_{k=1}^L p_k(\omega_i|x)\\
p(\omega_i|x)&=&\max\limits_i \sum\limits_{k=1}^L p_k(\omega_i|x)\\
p(\omega_i|x)&=&\max\limits_i \max\limits_{k\in\hat{L}} p_k(\omega_i|x)
\end{eqnarray}





\begin{equation}
Q=\sum\limits_{i=1}^N\sum\limits_{x\in C_i}\|x-m_{C_i}\|^2
\end{equation}


\ods{$N$-means clustering algoritmus}
\begin{enumerate}
\end{enumerate}



\begin{itemize}
\end{itemize}


\begin{eqnarray}
f&=&\min\limits_{a,b}\rho(a,b)\\
f&=&\max\limits_{a,b}\rho(a,b)\\
f&=&\rho(\mu_i,\mu_j)\\
\noalign{\hbox{{\em Hausdorfova metrika: }}}\nonumber\\ 
f&=&\max(\max\limits_a\rho(a,B),\max\limits_b\rho(b,A))
\end{eqnarray}




{\def\mez{\hskip0em}
\def\scal{0.35}
\def\mezv{\vskip12pt}
\begin{figure}[htbp]%
  \centerline{%
  \begin{tabular}{c}\scalebox{\scal}{\includepng{vys000_.ppm}}\\$iter= 0$\end{tabular}\mez%
  \begin{tabular}{c}\scalebox{\scal}{\includepng{vys001_.ppm}}\\$iter= 1$\end{tabular}\mez%
  \begin{tabular}{c}\scalebox{\scal}{\includepng{vys002_.ppm}}\\$iter= 2$\end{tabular}\mez%
  \begin{tabular}{c}\scalebox{\scal}{\includepng{vys003_.ppm}}\\$iter= 3$\end{tabular}\mez%
  }
   \mezv
  \centerline{%
  \begin{tabular}{c}\scalebox{\scal}{\includepng{vys004_.ppm}}\\$iter= 4$\end{tabular}\mez%
  \begin{tabular}{c}\scalebox{\scal}{\includepng{vys005_.ppm}}\\$iter= 5$\end{tabular}\mez%
  \begin{tabular}{c}\scalebox{\scal}{\includepng{vys006_.ppm}}\\$iter= 6$\end{tabular}\mez%
  \begin{tabular}{c}\scalebox{\scal}{\includepng{vys007_.ppm}}\\$iter= 7$\end{tabular}\mez%
   }\mezv%
  \centerline{%
  \begin{tabular}{c}\scalebox{\scal}{\includepng{vys008_.ppm}}\\$iter= 8$\end{tabular}\mez%
  \begin{tabular}{c}\scalebox{\scal}{\includepng{vys009_.ppm}}\\$iter= 9$\end{tabular}\mez%
  \begin{tabular}{c}\scalebox{\scal}{\includepng{vys010_.ppm}}\\$iter= 10$\end{tabular}\mez%
  \begin{tabular}{c}\scalebox{\scal}{\includepng{vys011_.ppm}}\\$iter= 11$\end{tabular}\mez%
   }%
  \label{shlukpic}%
\end{figure}%
}




\begin{itemize}
\end{itemize}

\begin{enumerate}
\end{enumerate}

\ods{One-class}

\refobr{oneclass}.

\ods{Principle component transformation -- PCT (Karhunen, Loeve)}

\begin{enumerate}
\end{enumerate}


schopnosti}

\ods{Two-class a feature selection}

\begin{equation}
\Phi=\frac{(\mu_1-\mu_2)^2}{(\sigma_1^2+\sigma_2^2)}
\end{equation}

\def\bmu{\mathbf{\mu}}


\begin{eqnarray}
d_M=(\m_1-\m_2)(\Sigma_1+\Sigma_2)^{-1}(\m_1-\m_2)^T\\
d_B=\frac{1}{4}d_M+\frac{1}{2}\ln\frac{|\frac{1}{2}(\Sigma_1+\Sigma_2)|}{\sqrt{|\Sigma_1||\Sigma_2|}}
\end{eqnarray}
 

\ods{Algoritmus Branch \& Bound}
 
 



\ods{Algoritmus Sequentional forward selection (SFS)}
\begin{enumerate}
\item[$\vdots$]
\end{enumerate}


\ods{Algoritmus Plus $l$ minus $s$}

\subsection{Segmentace}
\begin{itemize}
\end{itemize}

\begin{enumerate}
\item Hrany
\end{enumerate}


\ods{ad 2) Hrany}



\begin{enumerate}
\item binarizace
\end{enumerate}





\ods{Popis hranice}




\begin{description}
\item[Dilatace $\oplus$ (expanze):] $A\oplus B=\{x|B_x\cap A\neq \emptyset\}$
\item[Eroze $\ominus$:] $A\ominus B=\{x|B_x\subset A\}$
\end{description}


\begin{equation}
\end{equation}

\begin{description}
\end{description}

\begin{description}
\end{description}

\def\R{\mathbf{R}}

\begin{description}
\end{description}

\begin{itemize}
\end{itemize}




\vecobr{0.8}{shapevec}{Konstrukce Shape-vectoru}


\begin{eqnarray}
{A'}_k&=&e^{-i\pi\phi}A_k\\
\noalign{\hbox{tedy}\nonumber}\\
|{A'}_k|&=&|e^{-i\pi\phi}||A_k|=|A_k|=\sqrt{\Re^2A_k+\Im^2A_k}
\end{eqnarray}

\begin{itemize}
\end{itemize}








{\em Momenty}.

\def\d{{\rm d}}
\begin{equation}
m_{pq}=\int x^py^qf(x,y)\d x\d y\qquad \hbox{pro $p,q=0,1,\dots$}
\end{equation}

a naopak.

\begin{eqnarray}
x_t&=&\frac{m_{10}}{m_{00}}\nonumber\\
y_t&=&\frac{m_{01}}{m_{00}}
\end{eqnarray}


\begin{equation}
\mu_{pq}=\int (x-x_t)^p(y-y_t)^qf(x,y)\d x\d y\qquad \hbox{pro $p,q=0,1,\dots$}
\end{equation}




\begin{eqnarray}
x'&=&ax\nonumber\\
y'&=&ay
\end{eqnarray}

Je pak ale
\begin{eqnarray}
{m'}_{pq}&=&\int {x'}^p{y'}^q{f'}(x',y')\d x'\d y'=\int (ax)^p(ay)^qf(x,y)a^2\d x\d y=\nonumber\\
&=&a^{p+q+2}\int x^py^qf(x,y)\d x\d y=a^{p+q+2} m_{pq}\\
\end{eqnarray}

\begin{eqnarray}
\nu_{pq}&=&\frac{\mu_{pq}}{\mu^\omega_{00}}\\
\noalign{\hbox{Pak bude}\nonumber}\\
{\nu'}_{pq}&=&\frac{{\mu'}_{pq}} {{\mu'}^\omega_{00}}=\frac{a^{p+q+2}\mu_{pq}}{a^{2\omega}\mu^\omega_{00}}=
\frac{a^{p+q+2}}{a^{2\omega}}\nu_{pq}\nonumber\\
\omega&=&\frac{p+q}{2}+1
\end{eqnarray}


\subsubsection{Rotace soustavy}
\begin{eqnarray}
x'&=&x\cos\phi-y\sin\phi\nonumber\\
y'&=&x\sin\phi+y\cos\phi
\end{eqnarray}

\begin{eqnarray}
{m'}_{20}&=&\int(x\cos\phi-y\sin\phi)^2f(x,y)\d x\d y\\
{m'}_{20}&=& m_{20}\cos^2\phi-2m_{11}\cos\phi\sin\phi +m_{02}\sin^2\phi \\
{m'}_{02}&=& m_{20}\sin^2\phi+2m_{11}\cos\phi\sin\phi +m_{02}\cos^2\phi 
\end{eqnarray}

\begin{equation}
{m'}_{20}+{m'}_{02}=m_{20}+m_{02}
\end{equation}
\begin{equation}
(m_{20}-m_{02})^2+4m^2_{11}
\end{equation}


\ods{Classical moments invariants}
\begin{eqnarray}
\Phi_1&=&\mu_{20}+\mu_{02}\\
\Phi_2&=&(\mu_{20}-\mu_{02})^2+4\mu^2_{11}\\
\Phi_3&=&(\mu_{30}-3\mu_{12})^2+(3\mu_{21}-\mu_{03})^2\\
\Phi_4&=&(\mu_{30}+\mu_{12})^2+(\mu_{21}+\mu_{03})^2\\
\Phi_5&=&(\mu_{30}-3\mu_{12})(\mu_{30}+\mu_{12}((\mu_{30}+\mu_{12})^2-3(\mu_{21}+\mu_{03})^2)+\nonumber\\
&&+(3\mu_{21}-\mu_{03})(\mu_{21}+\mu_{03})(3(\mu_{30}+\mu_{12})^2-(\mu_{21}+\mu_{03})^2)\\
\Phi_6&=&(\mu_{20}-\mu_{02})((\mu_{30}+\mu_{12})^2-(\mu_{21}+\mu_{03})^2)+\nonumber\\
&&+4\mu_{11}(\mu_{30}+\mu_{12})(\mu_{21}+\mu_{03})\\
\Phi_7&=&(3\mu_{21}-\mu_{03})(\mu_{30}+\mu_{12})((\mu_{30}+\mu_{12})^2-3(\mu_{21}+\mu_{03})^2)+\nonumber\\
&&-(\mu_{30}-3\mu_{12})(\mu_{21}+\mu_{03})(3(\mu_{30}+\mu_{12})^2-(\mu_{21}+\mu_{03})^2)
\end{eqnarray}

\begin{eqnarray}
C_{pq}&=&\int\limits_{-\infty}^\infty\int\limits_{-\infty}^\infty(x+iy)^p(x-iy)^q f(x,y)\d x\d y\\
C_{pq}&=&\int\limits_0^\infty\int\limits_0^{2\phi}r^{p+q}e^{i(p-q)\theta}\tilde{f}(r,\theta)\d r\d \theta
\end{eqnarray}
\begin{equation}
{C'}_{pq}=C_{pq}\cdot e^{-i(p-q)\phi}
\end{equation}

\begin{eqnarray}
I&=&\prod\limits_{j=1}^nC_{p_jq_j}^{k_j}\\
\sum\limits_{j=1}^nk_j(p_j-q_j)&=&0
\end{eqnarray}
\begin{eqnarray}
C_{00}&=&m_{00}\\
C_{10}&=&m_{10}+im_{01}\\
C_{20}&=&m_{20}-m_{02}+2im_{11}\\
C_{11}&=&m_{20}+m_{02}
\end{eqnarray}

\begin{equation}
m_{pq}=\sum\limits_{ij}i^pj^qf_{ij}
\end{equation}

\begin{eqnarray}
m_p&=&\int x^p f(x)\d x=\sum\limits_{i}f_i\int\limits_{A_i}x^p\d x\quad\hbox{kde $A_i\equiv<i,i+1>$}\\
&=&\sum\limits_{i}f_i\left[\frac{x^{p+1}}{p+1}\right]^{i+1}_i=\sum\limits_{i}f_i
\left[\frac{(i+1)^{p+1}}{p+1}-\frac{i^{p+1}}{p+1}\right]
\end{eqnarray}


\begin{itemize}
\end{itemize}


\begin{eqnarray}
F(u)&=&\int e^{-iux}f(x)\d x=\int\sum\frac{(-iux)^n}{n!}f(x)\d x=\nonumber\\
&=&\sum\frac{(-i)^n}{n!}u^n\int x^nf(x)=\sum\frac{(-i)^n}{n!}u^n m_n\label{rada}
\end{eqnarray}




\begin{eqnarray}
G&=&F\cdot H\\
|G|&=&|F||H|\\
ph(G)&=&ph(F)+ph(H)\\
ph(H)&=&\left\{\begin{array}{ll}0\\\pi\end{array}\right.\\
\tan ph(G)&=&\tan ph(F)
\end{eqnarray}

\begin{eqnarray}
\mu_{20}^g&=&\mu_{20}^f\mu_{00}^h+\mu_{20}^h\mu_{00}^f\\
\mu_{02}^g&=&\mu_{02}^f\mu_{00}^h+\mu_{02}^h\mu_{00}^f\\
\mu_{20}^g-\mu_{02}^g&=&(\mu_{20}^f-\mu_{02}^f)\mu_{00}^h+(\mu_{20}^h-\mu_{02}^h)\mu_{00}^f\\
\mu_{20}^g-\mu_{02}^g&=&\mu_{20}^f-\mu_{02}^f
\end{eqnarray}

\section*{Disclaimer}
{
\sc
vrub.

}

